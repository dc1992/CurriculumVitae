\documentclass[margin, 10pt]{res} % Use the res.cls style, the font size can be changed to 11pt or 12pt here
\usepackage{helvet} 
\setlength{\textwidth}{5.1in} % Text width of the document

\begin{document}

\moveleft.5\hoffset\centerline{\large\bf Diego Ceccacci}
 
\moveleft\hoffset\vbox{\hrule width\resumewidth height 1pt}\smallskip % Horizontal line after name; adjust line thickness by changing the '1pt'
 
\moveleft.5\hoffset\centerline{Curriculum vitae et studiorum}

\begin{resume}

\section{Dati personali}

{\sl Luogo e data di nascita:} \hspace{0.9mm} Jesi (AN), 23/07/1992 \\
{\sl Residenza:} \hspace{21.8mm} Via Tommasi 7/b, Falconara Marittima (AN) \\
{\sl Domicilio:} \hspace{22.5mm} Vicolo Roggiuzzole 22, Pordenone (PN) \\
{\sl Recapito telefonico:} \hspace{8mm} 3398244709 \\
{\sl Indirizzo e-mail:} \hspace{13.2mm} diegoceccacci92@gmail.com \\
{\sl LinkedIn:} \hspace{23.5mm} https://www.linkedin.com/in/diego-ceccacci/ \\
{\sl GitHub:} \hspace{25.5mm} https://github.com/dc1992 \\

\section{Istruzione e formazione}

{\sl \textbf{Laurea in Informatica Applicata}}  \hfill Ottobre 2012 - Settembre 2016 \\
Universit\`a degli studi di Urbino "Carlo Bo", Urbino (PU)
\begin{itemize}
	\item Tesi di Laurea: "Reingegnerizzazione di un sistema informatico con architettura a microservizi"
\end{itemize}

{\sl \textbf{Diploma di Liceo Scientifico Tecnologico}}  \hfill Settembre 2006 - Luglio 2011 \\
Istituto di Istruzione Superiore "Volterra Elia", Ancona (AN) \\
 
\section{Esperienze lavorative}

{\sl \textbf{Sviluppatore Software}} \hfill Settembre 2021 - Oggi \\
Flowing, Full Remote \\
Mi occupo di raccolta requisiti, sviluppo e manutenzione di software per diversi clienti.

{\sl \textbf{Sviluppatore Software .NET}} \hfill Febbraio 2016 - Agosto 2021 \\
PhotoS\`i  S.p.a., Riccione (RN) \\
Mi sono occupato dello sviluppo e manutenzione di software per il sistema informatico dell'azienda. In particolare le mie attivit\`a comprendono bug fixing, analisi e implementazione di nuove funzionalit\`a richieste, refactoring di applicativi legacy utilizzando tecnologie innovative, scrittura di test unitari e funzionali e ottimizzazione codice preesistente.

{\sl \textbf{Sviluppatore Software}} \hfill Giugno 2018 - Settembre 2018 \\
Green Dreams, Pesaro (PU) \\
Ho collaborato sviluppando, da remoto, software per diversi progetti a cui l'azienda stava lavorando (ho utilizzato Java e C\texttt{\#}).

{\sl \textbf{Prestazione lavoro accessorio}} \hfill Febbraio 2013 \\
Associazione culturale NeuNet, Urbino (PU) 
\begin{itemize} %\itemsep -2pt % Reduce space between items
\item Incarico con Contratto di Prestazione Lavoro Accessorio per la partecipazione al Progetto dell'Associazione Culturale NeuNet di Urbino, al progetto "Descrizione strategie di gioco adottate in Timeville". \\
\end{itemize}


\section{Competenze tecniche}

\begin{itemize}

\item {\sl Linguaggi:} Conoscenza approfondita di C\texttt{\#} (.NET 4.5.2, .NET Core 2.2 e ASP .NET Core), Sql, php e Javascript (Node.js e Typescript). Conoscenza base di Html, Java, C, xml, Visual Basic, Python e \LaTeX.
\item {\sl Framework principali:} Entity Framework (6 e core), LinQ, NUnit, RazorEngine, Autofac, Express, Next.js.
\item {\sl Strumenti:} Visual Studio, Git, SQL Server, PostgreSQL e padronanza dei principali sistemi operativi.
\item {\sl Ingegneria del software:} Scrum, Continuous Integration, TDD e i principali design pattern.

\end{itemize}

\section{Competenze personali}

Sono curioso e trovo stimolante imparare e sperimentare l'utilizzo di nuovi framework, linguaggi e metodologie di sviluppo. Cerco ogni giorno di migliorare il modo in cui scrivo codice in modo da renderlo piacevole da leggere e facile da modificare. Anche se prediligo lo sviluppo backend, mi piacerebbe fare esperienza con progetti frontend. Ho trovato particolarmente interessante il libro "Clean Code" di Robert C. Martin.

\section{Altro}

Automunito (patente A e B), buona conoscenza della lingua inglese parlata e scritta (livello B1) e appassionato di nuove tecnologie, soprattutto quelle riguardanti il mondo del mobile e dell'intrattenimento videoludico.

\end{resume}
\end{document}